\section{Evaluación}
  \subsection{Bonificaciones}
    Además del puntaje asignado por cumplir con los requisitos anteriores, puede recibir 
    puntos adicionales (\emph{que se sumarán a su nota total}) implementando lo siguiente:
    
    \begin{itemize}
      \item \textbf{Mini-tareas} (0.4 pts.): De acuerdo a las entregas que haya hecho de las 
        mini-tareas puede recibir una bonificación de hasta 0.4 pts.
        Entregue las mini-tareas solamente si considera que están completas o altamente completas.
        Entregar mini-tareas demasiado incompletas puede significar un descuento en el puntaje.
        Para los casos de mini-tareas que estén completas pero contengan errores pueden optar a una
        fracción del puntaje.
      %
      \item \textbf{Exclusivamente para la tarea 3:}
      
      \begin{itemize}
        %
        \item \textbf{Interfaz gráfica avanzada} (0.5 pts.): Si su interfaz gráfica cumple 
          más de los requisitos mínimos podrá recibir una bonificación de hasta 0.5 pts. 
          (esto quedará a criterio del ayudante que le revise).
        %
        \item \textbf{Manejo de excepciones} (0.3 pts.): Se otorgará puntaje por la correcta 
          utilización de excepciones para manejar casos de borde en el juego.
          Recuerde que atrapar \textit{runtime exceptions} es una mala práctica, así como
          arrojar un error de tipo \java{Exception} (esto último ya que no es un error lo 
          suficientemente descriptivo).
          Bajo ninguna circunstancia una de estas excepciones debiera llegar al usuario.
      \end{itemize}
      %
    \end{itemize}
  %
  \subsection{Requisitos adicionales}
    Que su programa funcione no es suficiente, se espera además que éste presente un buen 
    diseño, siendo esta la característica a la que se le dará \textbf{mayor importancia} 
    en la revisión de sus entregas.

    Sus programas además deberán estar bien testeados, por lo que \textbf{NO SE REVISARÁN}
    las funcionalidades que no tengan un test correspondiente que pruebe su correcto 
    funcionamiento.
    Para revisar esto, también es de suma importancia que el trabajo que entregue pueda 
    ejecutarse (que su código compile), en caso contrario no podrá revisarse la 
    funcionalidad y por ende no tendrá puntaje en este aspecto.

    Otro aspecto que se tendrá en cuenta al momento de revisar sus tareas es que estas 
    estén bien documentadas, siguiendo los estándares de documentación para 
    \textit{Java} de \textit{Google}.\footnote{
      \url{https://github.com/CC3002-Metodologias/apunte-y-ejercicios/wiki/Convenciones}
    }

    Todo su trabajo debe ser subido al repositorio privado que se le entregó a través de 
    \textit{Github Classroom}.
    La modalidad de entrega será mediante un resumen en \textbf{formato PDF} entregado 
    mediante \textit{u-cursos} que contenga su nombre, \textit{rut}, un diagrama de clases de su 
    programa y un enlace a su repositorio (no se aceptará código recibido por este 
    medio).\footnote{No cumplir con estas condiciones implicará descuento de puntaje.}
    Además su repositorio deberá contener un archivo \texttt{README.md}\footnotemark que 
    contenga todas las instrucciones necesarias para ejecutar su programa, todos los 
    supuestos que realice y una breve explicación del funcionamiento y la lógica de su 
    programa.
    \footnotetext{
      \url{https://help.github.com/en/articles/basic-writing-and-formatting-syntax}
    }

    Para la tarea deberá crear su propia rama para trabajar y subir todo su código en esta y, cuando
    tenga una entrega lista para ser revisada, realizar un \textbf{\textit{pull request}}\footnote{
      \url{
        https://help.github.com/en/github/collaborating-with-issues-and-pull-requests/creating-a-pull-request
      }
    }, esta será la versión que será revisada.
    Tenga en cuenta que si hace \textit{push} de otros \textit{commits} en esa rama se actualizará 
    el \textit{PR}, por lo que se le recomienda crear una nueva rama para seguir trabajando.    
  %

  \subsection{Plazos de entrega}
    No se aceptarán peticiones de extensión de plazo, pero dispondrá de 72 horas a lo largo
    del semestre para entregar tareas atrasadas sin que se aplique descuento.
    Esto significa que si la entrega de la primera tarea se hace con 72 horas de atraso, 
    entonces las siguientes tendrán que entregarse sin atraso.
    No es necesario dar aviso al cuerpo docente para usar las horas, simplemente se 
    revisará el \textit{commit} correspondiente a la entrega de acuerdo a su \textit{Pull request}.
    
    Las horas de atraso serán aproximadas: para esto, se considerará cada media 
    hora de atraso como 1 hora, y menos que eso como 0 horas, exceptuando la primera hora.
    Entonces, si se entrega la tarea con 1 minuto de atraso, cuenta como 1 hora de atraso;
    si se entrega con 1 hora y 1 minuto, también contará como 1 hora; si se entrega con 
    1:30 de atraso se contará como 2 horas, y así.

    La entrega de las mini-tareas se hará también mediante \textit{u-cursos}, para esto basta 
    entregar un documento de texto plano con su nombre, \textit{rut}, y un enlace al 
    \textit{Pull request} que cuente como entrega. 
  %
  \subsection{Evaluación}
    \begin{itemize}
      \item \textbf{Código fuente (4.0 puntos):} Este ítem se divide en 2:
        \begin{itemize}
          \item \textbf{Funcionalidad (1.5 puntos):} Se analizará que su código provea la 
            funcionalidad pedida.
            Para esto, se exigirá que testee las funcionalidades que 
            implementó\footnotemark.
            \textbf{Si una funcionalidad no se testea, no se podrá comprobar 
            que funciona y, por lo tanto, NO SE ASIGNARÁ PUNTAJE por ella.}
          \item \textbf{Diseño (2.5 puntos):} Se analizará que su código provea la 
            funcionalidad implementada utilizando un buen diseño.
        \end{itemize}
        \footnotetext{
          Se le recomienda enfáticamente que piense en cuales son los casos de borde de su
          implementación y en las fallas de las que podría aprovecharse un usuario
          malicioso.
        }
      \item \textbf{\textit{Coverage} (1.0 puntos):} Sus casos de prueba deben crear 
        diversos escenarios y contar con un \textit{coverage} de las líneas de al menos 
        90\% por paquete. 
        No está de más decir que sus tests deben testear algo (es decir, no ser 
        tests vacíos o sin \textit{asserts}).
      \item \textbf{\textit{Javadoc} (0.5 puntos):} Cada clase, interfaz y método público 
        debe ser debidamente documentado. 
        Se descontará por cada falta.
      \item \textbf{Resumen (0.5 puntos):} El resumen mencionado en la sección 
        anterior. 
        Se evaluará que haya incluido el diagrama UML de su proyecto. 
        \textbf{En caso de no enviarse el resumen con el link a su repositorio su 
        tarea no será corregida}.\footnote{Porque no tenemos su código.}
    \end{itemize}
  %
  \subsection{Recomendaciones}
    Como se mencionó anteriormente, no es suficiente que su tarea funcione correctamente. 
    Este curso contempla el diseño de su solución y la aplicación de buenas prácticas 
    de programación que ha aprendido en el curso. 
    Dicho esto, no se conforme con el primer diseño que se le venga a la mente, intente 
    ver si puede mejorarlo y hacerlo más extensible.

    \textbf{No comience su tarea a último momento}. 
    Esto es lo que se dice en todos los cursos, pero es particularmente importante/cierto 
    en este. 
    Si usted hace la tarea a último minuto lo más seguro es que no tenga tiempo para 
    reflexionar sobre su diseño, y termine entregando un diseño deficiente o sin usar 
    lo enseñado en el curso.
    
    Haga la documentación de su programa en inglés (no es necesario). 
    La documentación de casi cualquier programa \textit{open-source} se encuentra en este 
    idioma. 
    Considere esta oportunidad para practicar su inglés.
    
    Se les pide encarecidamente que las consultas referentes a la tarea las hagan por 
    el \textbf{foro de U-Cursos}. 
    En caso de no obtener respuesta en un tiempo razonable, pueden hacerle llegar 
    un correo al auxiliar o ayudantes.

    Por último, el orden en el que escriben su programa es importante, se le sugiere que 
    para cada funcionalidad que quiera implementar:
    \begin{enumerate}
      \item Cree los \textit{tests} necesarios para verificar la funcionalidad deseada, de
        esta manera el enfoque está en como debería funcionar ésta, y no en cómo debería
        implementarse.
        Esto es muy útil para pensar bien en cuál es el problema que se está buscando 
        resolver y se tengan presentes cuales serían las condiciones de borde que podrían 
        generar problemas para su implementación.
      %
      \item Escriba la firma y la documentación del método a implementar, de esta forma se
        tiene una definición de lo que hará su método incluso antes de implementarlo y se 
        asegura de que su programa esté bien documentado.
        Además, esto hace más entendible el código no solo para alguna persona que revise 
        su programa, sino que también para el mismo programador\footnotemark.
      %
      \item Por último implemente la funcionalidad pensando en que debe pasar los tests que
        escribió anteriormente y piense si estos tests son suficientes para cubrir todos 
        los escenarios posibles para su aplicación, vuelva a los pasos 1 y 2 si es 
        necesario.
    \end{enumerate}
    \footnotetext{
      Entender un programa mal documentado que haya escrito uno mismo después de varios 
      días de no trabajar en él puede resultar bastante complicado.
    }
  %
%