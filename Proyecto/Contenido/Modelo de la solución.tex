\section{Modelo de la solución}
  Para guiar la solución de este problema, se plantearan objetivos pequeños en forma de 
  \textit{mini-tareas} que buscarán enfrentar una problemática a la vez.
  Dichas tareas tendrán fecha de entrega pero será opcional entregar en la fecha señalada y no serán
  evaluadas con nota.
  Solamente será evaluado que al momento de entregar la tarea se cumplan todos los objetivos 
  planteados en las \textit{mini-tareas}.

  La resolución de este proyecto se hará siguiendo el patrón arquitectónico 
  \textit{Modelo-Vista-Controlador}\footnotemark, donde primero se implementará el 
  \textit{modelo}, luego el \textit{controlador} y por último la \textit{vista}.
  Este patrón se explicará en más detalle en el transcurso del curso, pero en el 
  contexto del proyecto estos componentes serán como se explica a continuación.
  \footnotetext{
    \url{https://www.github.com/CC3002-Metodologias/apunte-y-ejercicios/wiki/Modelo-Vista-Controlador}
  }

  \paragraph{Modelo}
    Para la primera parte se le solicitará que cree todas las entidades necesarias que
    servirán de estructura base del proyecto y las interacciones posibles entre dichas
    entidades.
    Las entidades en este caso se refieren a los elementos que componen el juego.
  %

  \paragraph{Vista}
    Se le pedirá también que cree una interfaz gráfica simple para el juego que pueda 
    responder al input de un usuario y mostrar toda la información relevante del juego en
    pantalla.
  %

  \paragraph{Controlador}
    Servirá de conexión lógica entre la vista y el modelo, se espera que el controlador 
    pueda ejecutar todas las operaciones que un jugador podría querer efectuar, que 
    entregue los mensajes necesarios a cada objeto del modelo y que guarde la 
    información más importante del estado del juego en cada momento.
  %
%