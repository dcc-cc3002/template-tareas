\section{Requisitos}
  
  \subimport{Requisitos/}{Código base.tex}

  \subimport{./Requisitos/}{Distancias.tex}
  \subimport{./Requisitos/}{State.tex}
  \subimport{./Requisitos/}{GUI.tex}

  \subsection{Bonificaciones}
    Además del puntaje asignado por cumplir con los requisitos anteriores, puede recibir 
    puntos adicionales (\emph{que se sumarán a su nota total}) implementando lo siguiente:
    \begin{itemize}
      \item \textbf{Interfaz gráfica avanzada} (0.5 pts): Si su interfaz gráfica cumple 
        más de los requisitos mínimos podrá recibir una bonificación de hasta 0.5 pts 
        (esto quedará a criterio del ayudante que le revise).
      \item \textbf{Manejo de excepciones} (0.3 pts): Se otorgará puntaje por la correcta 
        utilización de excepciones para manejar casos de borde en el juego.
        Recuerde que atrapar \textit{runtime exceptions} es una mala práctica, así como
        arrojar un error de tipo \java{Exception} (esto último ya que no es un error lo 
        suficientemente descriptivo).
        Bajo ninguna circunstancia una de estas excepciones debiera llegar al usuario.
      \item \textbf{\textit{Benchmarking}} (0.3 pts): Cree \textit{tests} que muestren que 
        su implementación del cálculo de distancias efectivamente mejora la eficiencia del
        programa considerando la implementación original del algoritmo y las 2 
        optimizaciones presentadas en este documento además de la que usted implementó.
        Agregue al \textit{readme} los resultados de dichos \textit{tests}.
    \end{itemize}
  %
  
  \subsection{Requerimientos adicionales}
    Además de una implementación basada en las buenas prácticas y técnicas de diseño 
    vistas en clases, usted debe considerar:
    \begin{itemize}
      \item \textbf{Cobertura:} Cree los \textit{tests} unitarios, usando JUnit 5, que sean 
        necesarios para tener al menos un coverage del 90 \% de las líneas por 
        paquete.
        Para esta entrega no debe testear las interfaces gráficas dado que no posee las 
        herramientas necesarias para eso, así que debe asegurarse de que las interfaces 
        gráficas estén en su propio paquete.
        El porcentaje de coverage se medirá respecto al resto de los paquetes.
        Todos los \textit{tests} de su proyecto deben estar en el paquete \java{test}.
      \item \textbf{Javadoc:} Cada interfaz, clase pública y método público deben 
        estar debidamente documentados siguiendo las convenciones de 
        Javadoc\footnotemark. 
        En particular necesita \java{@author} y una pequeña descripción para su 
        clase e interfaz, y \java{@param}, \java{@return} (si aplica) y una 
        descripción para los métodos.
        \footnotetext{
          {\url{http://www.oracle.com/technetwork/articles/java/index-137868.html}}
        }
      \item \textbf{Resumen:} Debe entregar un archivo \textbf{pdf} que contenga su 
        nombre, rut, usuario de Github, un link al repositorio de su tarea y un 
        diagrama UML que muestre las clases, interfaces y métodos que usted 
        definió en su proyecto. 
        \textbf{No debe incluir los tests} en su diagrama.
      \item \textbf{Git:} Debe hacer uso de git para el versionamiento de su 
        proyecto. 
        Luego, esta historia de versionamiento debe ser subida a Github.
      \item \textbf{Readme:} Debe hacer un \textit{readme} especificando los detalles de
        su implementación, los supuestos que realice y una breve explicación de cómo 
        ejecutar el programa.
        Adicionalmente se le solicita dar una explicación general de la estructura que 
        decidió utilizar, los patrones de diseño y la razón por la cual los utiliza.
    \end{itemize}

    Además debe considerar los requisitos que se especifican en el resumen del proyecto.
  %
%